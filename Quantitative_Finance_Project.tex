% Options for packages loaded elsewhere
\PassOptionsToPackage{unicode}{hyperref}
\PassOptionsToPackage{hyphens}{url}
%
\documentclass[
]{article}
\usepackage{amsmath,amssymb}
\usepackage{iftex}
\ifPDFTeX
  \usepackage[T1]{fontenc}
  \usepackage[utf8]{inputenc}
  \usepackage{textcomp} % provide euro and other symbols
\else % if luatex or xetex
  \usepackage{unicode-math} % this also loads fontspec
  \defaultfontfeatures{Scale=MatchLowercase}
  \defaultfontfeatures[\rmfamily]{Ligatures=TeX,Scale=1}
\fi
\usepackage{lmodern}
\ifPDFTeX\else
  % xetex/luatex font selection
\fi
% Use upquote if available, for straight quotes in verbatim environments
\IfFileExists{upquote.sty}{\usepackage{upquote}}{}
\IfFileExists{microtype.sty}{% use microtype if available
  \usepackage[]{microtype}
  \UseMicrotypeSet[protrusion]{basicmath} % disable protrusion for tt fonts
}{}
\makeatletter
\@ifundefined{KOMAClassName}{% if non-KOMA class
  \IfFileExists{parskip.sty}{%
    \usepackage{parskip}
  }{% else
    \setlength{\parindent}{0pt}
    \setlength{\parskip}{6pt plus 2pt minus 1pt}}
}{% if KOMA class
  \KOMAoptions{parskip=half}}
\makeatother
\usepackage{xcolor}
\usepackage[margin=1in]{geometry}
\usepackage{color}
\usepackage{fancyvrb}
\newcommand{\VerbBar}{|}
\newcommand{\VERB}{\Verb[commandchars=\\\{\}]}
\DefineVerbatimEnvironment{Highlighting}{Verbatim}{commandchars=\\\{\}}
% Add ',fontsize=\small' for more characters per line
\usepackage{framed}
\definecolor{shadecolor}{RGB}{248,248,248}
\newenvironment{Shaded}{\begin{snugshade}}{\end{snugshade}}
\newcommand{\AlertTok}[1]{\textcolor[rgb]{0.94,0.16,0.16}{#1}}
\newcommand{\AnnotationTok}[1]{\textcolor[rgb]{0.56,0.35,0.01}{\textbf{\textit{#1}}}}
\newcommand{\AttributeTok}[1]{\textcolor[rgb]{0.13,0.29,0.53}{#1}}
\newcommand{\BaseNTok}[1]{\textcolor[rgb]{0.00,0.00,0.81}{#1}}
\newcommand{\BuiltInTok}[1]{#1}
\newcommand{\CharTok}[1]{\textcolor[rgb]{0.31,0.60,0.02}{#1}}
\newcommand{\CommentTok}[1]{\textcolor[rgb]{0.56,0.35,0.01}{\textit{#1}}}
\newcommand{\CommentVarTok}[1]{\textcolor[rgb]{0.56,0.35,0.01}{\textbf{\textit{#1}}}}
\newcommand{\ConstantTok}[1]{\textcolor[rgb]{0.56,0.35,0.01}{#1}}
\newcommand{\ControlFlowTok}[1]{\textcolor[rgb]{0.13,0.29,0.53}{\textbf{#1}}}
\newcommand{\DataTypeTok}[1]{\textcolor[rgb]{0.13,0.29,0.53}{#1}}
\newcommand{\DecValTok}[1]{\textcolor[rgb]{0.00,0.00,0.81}{#1}}
\newcommand{\DocumentationTok}[1]{\textcolor[rgb]{0.56,0.35,0.01}{\textbf{\textit{#1}}}}
\newcommand{\ErrorTok}[1]{\textcolor[rgb]{0.64,0.00,0.00}{\textbf{#1}}}
\newcommand{\ExtensionTok}[1]{#1}
\newcommand{\FloatTok}[1]{\textcolor[rgb]{0.00,0.00,0.81}{#1}}
\newcommand{\FunctionTok}[1]{\textcolor[rgb]{0.13,0.29,0.53}{\textbf{#1}}}
\newcommand{\ImportTok}[1]{#1}
\newcommand{\InformationTok}[1]{\textcolor[rgb]{0.56,0.35,0.01}{\textbf{\textit{#1}}}}
\newcommand{\KeywordTok}[1]{\textcolor[rgb]{0.13,0.29,0.53}{\textbf{#1}}}
\newcommand{\NormalTok}[1]{#1}
\newcommand{\OperatorTok}[1]{\textcolor[rgb]{0.81,0.36,0.00}{\textbf{#1}}}
\newcommand{\OtherTok}[1]{\textcolor[rgb]{0.56,0.35,0.01}{#1}}
\newcommand{\PreprocessorTok}[1]{\textcolor[rgb]{0.56,0.35,0.01}{\textit{#1}}}
\newcommand{\RegionMarkerTok}[1]{#1}
\newcommand{\SpecialCharTok}[1]{\textcolor[rgb]{0.81,0.36,0.00}{\textbf{#1}}}
\newcommand{\SpecialStringTok}[1]{\textcolor[rgb]{0.31,0.60,0.02}{#1}}
\newcommand{\StringTok}[1]{\textcolor[rgb]{0.31,0.60,0.02}{#1}}
\newcommand{\VariableTok}[1]{\textcolor[rgb]{0.00,0.00,0.00}{#1}}
\newcommand{\VerbatimStringTok}[1]{\textcolor[rgb]{0.31,0.60,0.02}{#1}}
\newcommand{\WarningTok}[1]{\textcolor[rgb]{0.56,0.35,0.01}{\textbf{\textit{#1}}}}
\usepackage{graphicx}
\makeatletter
\def\maxwidth{\ifdim\Gin@nat@width>\linewidth\linewidth\else\Gin@nat@width\fi}
\def\maxheight{\ifdim\Gin@nat@height>\textheight\textheight\else\Gin@nat@height\fi}
\makeatother
% Scale images if necessary, so that they will not overflow the page
% margins by default, and it is still possible to overwrite the defaults
% using explicit options in \includegraphics[width, height, ...]{}
\setkeys{Gin}{width=\maxwidth,height=\maxheight,keepaspectratio}
% Set default figure placement to htbp
\makeatletter
\def\fps@figure{htbp}
\makeatother
\setlength{\emergencystretch}{3em} % prevent overfull lines
\providecommand{\tightlist}{%
  \setlength{\itemsep}{0pt}\setlength{\parskip}{0pt}}
\setcounter{secnumdepth}{-\maxdimen} % remove section numbering
\usepackage{pdfpages}
\usepackage{graphicx}
\ifLuaTeX
  \usepackage{selnolig}  % disable illegal ligatures
\fi
\IfFileExists{bookmark.sty}{\usepackage{bookmark}}{\usepackage{hyperref}}
\IfFileExists{xurl.sty}{\usepackage{xurl}}{} % add URL line breaks if available
\urlstyle{same}
\hypersetup{
  pdftitle={Quantitative Finance Project},
  pdfauthor={José Thiéry M. Hagbe},
  hidelinks,
  pdfcreator={LaTeX via pandoc}}

\title{Quantitative Finance Project}
\author{José Thiéry M. Hagbe}
\date{15 février, 2024}

\begin{document}
\maketitle

{
\setcounter{tocdepth}{2}
\tableofcontents
}
\newpage

\hypertarget{download-apple-daily-data-from-yahoo-finance-with-prices-for-one-stock-from-2019-01-01-to-2024-01-01.}{%
\section{1. Download APPLE daily data from Yahoo Finance with prices for
one stock from 2019-01-01, to
2024-01-01.}\label{download-apple-daily-data-from-yahoo-finance-with-prices-for-one-stock-from-2019-01-01-to-2024-01-01.}}

\begin{Shaded}
\begin{Highlighting}[]
\FunctionTok{library}\NormalTok{(quantmod)}
\CommentTok{\# Download APPLE daily data from Yahoo Finance}
\FunctionTok{getSymbols}\NormalTok{(}\StringTok{"AAPL"}\NormalTok{, }\AttributeTok{src =} \StringTok{"yahoo"}\NormalTok{, }\AttributeTok{from =} \StringTok{"2019{-}01{-}01"}\NormalTok{, }\AttributeTok{to =} \StringTok{"2024{-}01{-}01"}\NormalTok{)}
\end{Highlighting}
\end{Shaded}

\begin{verbatim}
## [1] "AAPL"
\end{verbatim}

\begin{Shaded}
\begin{Highlighting}[]
\NormalTok{Daily }\OtherTok{\textless{}{-}} \StringTok{\textasciigrave{}}\AttributeTok{AAPL}\StringTok{\textasciigrave{}}
\FunctionTok{colnames}\NormalTok{(Daily) }\OtherTok{\textless{}{-}} \FunctionTok{c}\NormalTok{(}\StringTok{"Open"}\NormalTok{, }\StringTok{"High"}\NormalTok{, }\StringTok{"Low"}\NormalTok{, }\StringTok{"Close"}\NormalTok{, }\StringTok{"Volume"}\NormalTok{, }\StringTok{"Adjusted"}\NormalTok{)}
\end{Highlighting}
\end{Shaded}

\hypertarget{display-the-time-series-plot-of-the-daily-close-data.-based-on-this-information-do-these-data-appear-to-come-from-a-stationary-or-nonstationary-process}{%
\section{2. Display the time series plot of the daily close data. Based
on this information, do these data appear to come from a stationary or
nonstationary
process?}\label{display-the-time-series-plot-of-the-daily-close-data.-based-on-this-information-do-these-data-appear-to-come-from-a-stationary-or-nonstationary-process}}

\begin{Shaded}
\begin{Highlighting}[]
\CommentTok{\# Create daily close prices time series}
\NormalTok{X }\OtherTok{\textless{}{-}} \FunctionTok{ts}\NormalTok{(}\AttributeTok{data =}\NormalTok{ Daily}\SpecialCharTok{$}\NormalTok{Close, }\AttributeTok{start =} \FunctionTok{c}\NormalTok{(}\DecValTok{2019}\NormalTok{, }\DecValTok{1}\NormalTok{, }\DecValTok{1}\NormalTok{), }\AttributeTok{end =}  \FunctionTok{c}\NormalTok{(}\DecValTok{2024}\NormalTok{, }\DecValTok{1}\NormalTok{, }\DecValTok{1}\NormalTok{), }\AttributeTok{frequency =} \DecValTok{365}\NormalTok{)}

\CommentTok{\# Create daily close prices time series plot}
\FunctionTok{tsplot}\NormalTok{(X, }\AttributeTok{main =} \StringTok{"Apple daily close prices time series"}\NormalTok{, }\AttributeTok{lwd =} \DecValTok{2}\NormalTok{) }
\end{Highlighting}
\end{Shaded}

\includegraphics{Quantitative_Finance_Project_files/figure-latex/unnamed-chunk-3-1.pdf}

\hypertarget{comment-it-appeared-that-these-data-are-coming-from-nonstationary-process}{%
\subsubsection{Comment: It appeared that these data are coming from
nonstationary
process}\label{comment-it-appeared-that-these-data-are-coming-from-nonstationary-process}}

\hypertarget{use-the-best-subsets-arima-approach-to-specify-a-model-for-the-daily-close-data.}{%
\section{3. Use the best subsets ARIMA approach to specify a model for
the daily close
data.}\label{use-the-best-subsets-arima-approach-to-specify-a-model-for-the-daily-close-data.}}

\hypertarget{decomposing-the-data}{%
\subsection{Decomposing the Data}\label{decomposing-the-data}}

Decomposing the data into its trend, seasonal, and random error
components will give some idea how these components relate to the
observed dataset.

\begin{Shaded}
\begin{Highlighting}[]
\CommentTok{\# Decomposition using multiplicative type }
\NormalTok{X.decomp }\OtherTok{\textless{}{-}} \FunctionTok{decompose}\NormalTok{(X, }\AttributeTok{type =} \StringTok{"multiplicative"}\NormalTok{)}
\FunctionTok{plot}\NormalTok{(X.decomp, }\AttributeTok{lwd=}\DecValTok{2}\NormalTok{)}
\end{Highlighting}
\end{Shaded}

\includegraphics{Quantitative_Finance_Project_files/figure-latex/unnamed-chunk-4-1.pdf}

\emph{Summary Statistics of the Data}

\begin{Shaded}
\begin{Highlighting}[]
\FunctionTok{summary}\NormalTok{(X)}
\end{Highlighting}
\end{Shaded}

\begin{verbatim}
##    Min. 1st Qu.  Median    Mean 3rd Qu.    Max. 
##   35.55   64.31  117.51  109.58  148.48  198.11
\end{verbatim}

\begin{Shaded}
\begin{Highlighting}[]
\FunctionTok{describe}\NormalTok{(X)}
\end{Highlighting}
\end{Shaded}

\begin{verbatim}
##    vars    n   mean   sd median trimmed   mad   min    max  range skew kurtosis
## X1    1 1826 109.58 46.8 117.51  108.65 63.62 35.55 198.11 162.56 0.03    -1.34
##     se
## X1 1.1
\end{verbatim}

\hypertarget{data-transformation-to-achieve-stationarity}{%
\subsubsection{Data Transformation To Achieve
Stationarity}\label{data-transformation-to-achieve-stationarity}}

Now, we will have to perform some data transformation to achieve
Stationarity.

\begin{Shaded}
\begin{Highlighting}[]
\CommentTok{\# Remove the trend from the data}
\CommentTok{\# First log Difference}
\NormalTok{d.X }\OtherTok{=} \FunctionTok{diff}\NormalTok{(}\FunctionTok{log}\NormalTok{(X)) }\CommentTok{\# to remove trend}
\FunctionTok{plot}\NormalTok{(d.X, }\AttributeTok{main =} \StringTok{"Log of Apple daily close prices time series"}\NormalTok{, }\AttributeTok{lwd=}\DecValTok{2}\NormalTok{)}
\end{Highlighting}
\end{Shaded}

\includegraphics{Quantitative_Finance_Project_files/figure-latex/unnamed-chunk-6-1.pdf}

\begin{Shaded}
\begin{Highlighting}[]
\NormalTok{d.logX}\FloatTok{.365} \OtherTok{=} \FunctionTok{diff}\NormalTok{(d.X, }\AttributeTok{lag =} \DecValTok{365}\NormalTok{)  }\CommentTok{\# remove seasonality}
\FunctionTok{plot}\NormalTok{(d.logX}\FloatTok{.365}\NormalTok{, }\AttributeTok{main =}\StringTok{" Log of Apple daily close price time series with lag = 365 "}\NormalTok{, }\AttributeTok{lwd=}\DecValTok{2}\NormalTok{, }\AttributeTok{col=}\StringTok{"blue"}\NormalTok{)}
\end{Highlighting}
\end{Shaded}

\includegraphics{Quantitative_Finance_Project_files/figure-latex/unnamed-chunk-6-2.pdf}

\begin{Shaded}
\begin{Highlighting}[]
\CommentTok{\# test for stationarity}
\FunctionTok{adf.test}\NormalTok{(d.logX}\FloatTok{.365}\NormalTok{)}
\end{Highlighting}
\end{Shaded}

\begin{verbatim}
## Warning in adf.test(d.logX.365): p-value smaller than printed p-value
\end{verbatim}

\begin{verbatim}
## 
##  Augmented Dickey-Fuller Test
## 
## data:  d.logX.365
## Dickey-Fuller = -10.636, Lag order = 11, p-value = 0.01
## alternative hypothesis: stationary
\end{verbatim}

\hypertarget{comment-we-see-that-the-series-is-stationary-enough-to-do-any-kind-of-time-series-modelling.}{%
\subsubsection{Comment: We see that the series is stationary enough to
do any kind of time series
modelling.}\label{comment-we-see-that-the-series-is-stationary-enough-to-do-any-kind-of-time-series-modelling.}}

\hypertarget{identification-of-the-best-model}{%
\subsection{Identification of the best
model}\label{identification-of-the-best-model}}

\begin{Shaded}
\begin{Highlighting}[]
\CommentTok{\# Model Selection}
\FunctionTok{set.seed}\NormalTok{(}\DecValTok{123}\NormalTok{)}
\FunctionTok{auto.arima}\NormalTok{(d.logX}\FloatTok{.365}\NormalTok{, }\AttributeTok{stationary =} \ConstantTok{TRUE}\NormalTok{, }\AttributeTok{ic =} \FunctionTok{c}\NormalTok{(}\StringTok{"aic"}\NormalTok{), }\AttributeTok{trace =} \ConstantTok{TRUE}\NormalTok{)}
\end{Highlighting}
\end{Shaded}

\begin{verbatim}
## 
##  Fitting models using approximations to speed things up...
## 
##  ARIMA(2,0,2)(1,0,1)[365] with non-zero mean : Inf
##  ARIMA(0,0,0)             with non-zero mean : -3800.64
##  ARIMA(1,0,0)(1,0,0)[365] with non-zero mean : Inf
##  ARIMA(0,0,1)(0,0,1)[365] with non-zero mean : Inf
##  ARIMA(0,0,0)             with zero mean     : -3802.639
##  ARIMA(0,0,0)(1,0,0)[365] with non-zero mean : Inf
##  ARIMA(0,0,0)(0,0,1)[365] with non-zero mean : Inf
##  ARIMA(0,0,0)(1,0,1)[365] with non-zero mean : Inf
##  ARIMA(1,0,0)             with non-zero mean : -3804.81
##  ARIMA(1,0,0)(0,0,1)[365] with non-zero mean : Inf
##  ARIMA(1,0,0)(1,0,1)[365] with non-zero mean : Inf
##  ARIMA(2,0,0)             with non-zero mean : -3805.562
##  ARIMA(2,0,0)(1,0,0)[365] with non-zero mean : Inf
##  ARIMA(2,0,0)(0,0,1)[365] with non-zero mean : Inf
##  ARIMA(2,0,0)(1,0,1)[365] with non-zero mean : Inf
##  ARIMA(3,0,0)             with non-zero mean : -3802.814
##  ARIMA(2,0,1)             with non-zero mean : -3804.158
##  ARIMA(1,0,1)             with non-zero mean : -3806.506
##  ARIMA(1,0,1)(1,0,0)[365] with non-zero mean : Inf
##  ARIMA(1,0,1)(0,0,1)[365] with non-zero mean : Inf
##  ARIMA(1,0,1)(1,0,1)[365] with non-zero mean : Inf
##  ARIMA(0,0,1)             with non-zero mean : -3802.139
##  ARIMA(1,0,2)             with non-zero mean : -3805.091
##  ARIMA(0,0,2)             with non-zero mean : -3803.142
##  ARIMA(2,0,2)             with non-zero mean : -3802.123
##  ARIMA(1,0,1)             with zero mean     : -3808.5
##  ARIMA(1,0,1)(1,0,0)[365] with zero mean     : Inf
##  ARIMA(1,0,1)(0,0,1)[365] with zero mean     : Inf
##  ARIMA(1,0,1)(1,0,1)[365] with zero mean     : Inf
##  ARIMA(0,0,1)             with zero mean     : -3804.139
##  ARIMA(1,0,0)             with zero mean     : -3806.806
##  ARIMA(2,0,1)             with zero mean     : -3806.112
##  ARIMA(1,0,2)             with zero mean     : -3807.087
##  ARIMA(0,0,2)             with zero mean     : -3805.142
##  ARIMA(2,0,0)             with zero mean     : -3807.56
##  ARIMA(2,0,2)             with zero mean     : -3804.178
## 
##  Now re-fitting the best model(s) without approximations...
## 
##  ARIMA(1,0,1)             with zero mean     : -3805.768
## 
##  Best model: ARIMA(1,0,1)             with zero mean
\end{verbatim}

\begin{verbatim}
## Series: d.logX.365 
## ARIMA(1,0,1) with zero mean 
## 
## Coefficients:
##           ar1     ma1
##       -0.6791  0.7307
## s.e.   0.1993  0.1859
## 
## sigma^2 = 0.004308:  log likelihood = 1905.88
## AIC=-3805.77   AICc=-3805.75   BIC=-3789.91
\end{verbatim}

\hypertarget{comment-auto-arima-indicates-us-that-the-best-arima-model-is-arima101}{%
\subsubsection{Comment: auto arima indicates us that the best arima
model is
ARIMA(1,0,1)}\label{comment-auto-arima-indicates-us-that-the-best-arima-model-is-arima101}}

\hypertarget{daily-model}{%
\subsection{Daily Model}\label{daily-model}}

\begin{Shaded}
\begin{Highlighting}[]
\DocumentationTok{\#\# Setting up the Model}
\NormalTok{Daily.model }\OtherTok{\textless{}{-}} \FunctionTok{arima}\NormalTok{(d.logX}\FloatTok{.365}\NormalTok{, }\AttributeTok{order =} \FunctionTok{c}\NormalTok{(}\DecValTok{1}\NormalTok{,}\DecValTok{0}\NormalTok{,}\DecValTok{1}\NormalTok{))}
\end{Highlighting}
\end{Shaded}

\hypertarget{model-evaluation}{%
\subsection{Model evaluation}\label{model-evaluation}}

\begin{Shaded}
\begin{Highlighting}[]
\DocumentationTok{\#\# Diagnostic Checking}
\NormalTok{U}\FloatTok{.1} \OtherTok{\textless{}{-}}\NormalTok{ Daily.model}\SpecialCharTok{$}\NormalTok{residuals}
\FunctionTok{plot}\NormalTok{(U}\FloatTok{.1}\NormalTok{)}
\end{Highlighting}
\end{Shaded}

\includegraphics{Quantitative_Finance_Project_files/figure-latex/unnamed-chunk-9-1.pdf}

\begin{Shaded}
\begin{Highlighting}[]
\CommentTok{\# Normality of the Residuals}
\FunctionTok{qqnorm}\NormalTok{(U}\FloatTok{.1}\NormalTok{); }\FunctionTok{qqline}\NormalTok{(U}\FloatTok{.1}\NormalTok{)}
\end{Highlighting}
\end{Shaded}

\includegraphics{Quantitative_Finance_Project_files/figure-latex/unnamed-chunk-9-2.pdf}

\begin{Shaded}
\begin{Highlighting}[]
\CommentTok{\# ACF of residuals}
\FunctionTok{acf}\NormalTok{(U}\FloatTok{.1}\NormalTok{)}
\end{Highlighting}
\end{Shaded}

\includegraphics{Quantitative_Finance_Project_files/figure-latex/unnamed-chunk-9-3.pdf}

\begin{Shaded}
\begin{Highlighting}[]
\FunctionTok{plot}\NormalTok{(}\FunctionTok{density}\NormalTok{(U}\FloatTok{.1}\NormalTok{))}
\end{Highlighting}
\end{Shaded}

\includegraphics{Quantitative_Finance_Project_files/figure-latex/unnamed-chunk-9-4.pdf}

\begin{Shaded}
\begin{Highlighting}[]
\CommentTok{\#Using tsdiag tools}
\FunctionTok{tsdiag}\NormalTok{(Daily.model, }\AttributeTok{gof.lag =} \DecValTok{15}\NormalTok{)}
\end{Highlighting}
\end{Shaded}

\includegraphics{Quantitative_Finance_Project_files/figure-latex/unnamed-chunk-9-5.pdf}

\hypertarget{forecasting}{%
\subsection{Forecasting}\label{forecasting}}

Using our best model, we can forecast the observations for the next 2
years.

\begin{Shaded}
\begin{Highlighting}[]
\NormalTok{Daily.pred }\OtherTok{\textless{}{-}} \FunctionTok{predict}\NormalTok{(Daily.model, }\AttributeTok{n.ahead =} \DecValTok{2}\SpecialCharTok{*}\DecValTok{365}\NormalTok{)}
\FunctionTok{ts.plot}\NormalTok{(X, }\FunctionTok{exp}\NormalTok{(Daily.pred}\SpecialCharTok{$}\NormalTok{pred), }\AttributeTok{log =} \StringTok{"y"}\NormalTok{, }\AttributeTok{lty =}\FunctionTok{c}\NormalTok{(}\DecValTok{1}\NormalTok{,}\DecValTok{3}\NormalTok{), }\AttributeTok{lwd =} \DecValTok{2}\NormalTok{)}
\end{Highlighting}
\end{Shaded}

\includegraphics{Quantitative_Finance_Project_files/figure-latex/unnamed-chunk-10-1.pdf}

\newpage

\hypertarget{use-the-best-subsets-arima-approach-to-specify-a-model-for-the-weekly-close-data.}{%
\section{4. Use the best subsets ARIMA approach to specify a model for
the weekly close
data.}\label{use-the-best-subsets-arima-approach-to-specify-a-model-for-the-weekly-close-data.}}

\begin{Shaded}
\begin{Highlighting}[]
\CommentTok{\# Aggregate daily data to weekly}
\NormalTok{Weekly }\OtherTok{\textless{}{-}} \FunctionTok{to.weekly}\NormalTok{(Daily)}
\FunctionTok{colnames}\NormalTok{(Weekly) }\OtherTok{\textless{}{-}} \FunctionTok{c}\NormalTok{(}\StringTok{"Open"}\NormalTok{, }\StringTok{"High"}\NormalTok{, }\StringTok{"Low"}\NormalTok{, }\StringTok{"Close"}\NormalTok{, }\StringTok{"Volume"}\NormalTok{, }\StringTok{"Adjusted"}\NormalTok{)}

\CommentTok{\# Create weekly close prices time series}
\NormalTok{Y }\OtherTok{\textless{}{-}} \FunctionTok{ts}\NormalTok{(}\AttributeTok{data =}\NormalTok{ Weekly}\SpecialCharTok{$}\NormalTok{Close, }\AttributeTok{start =} \FunctionTok{c}\NormalTok{(}\DecValTok{2019}\NormalTok{, }\DecValTok{1}\NormalTok{, }\DecValTok{1}\NormalTok{), }\AttributeTok{end =}  \FunctionTok{c}\NormalTok{(}\DecValTok{2024}\NormalTok{, }\DecValTok{1}\NormalTok{, }\DecValTok{1}\NormalTok{), }\AttributeTok{frequency =} \DecValTok{52}\NormalTok{)}

\CommentTok{\# Create weekly close prices time series}
\FunctionTok{plot}\NormalTok{(Y, }\AttributeTok{main=}\StringTok{"Apple Weekly close prices time series"}\NormalTok{, }\AttributeTok{lwd =} \DecValTok{2}\NormalTok{)}
\end{Highlighting}
\end{Shaded}

\includegraphics{Quantitative_Finance_Project_files/figure-latex/unnamed-chunk-11-1.pdf}

\hypertarget{decomposing-the-data-1}{%
\subsection{Decomposing the Data}\label{decomposing-the-data-1}}

Decomposing the data into its trend, seasonal, and random error
components will give some idea how these components relate to the
observed dataset.

\begin{Shaded}
\begin{Highlighting}[]
\CommentTok{\# Decomposition using multiplicative type }
\NormalTok{Y.decomp }\OtherTok{\textless{}{-}} \FunctionTok{decompose}\NormalTok{(Y, }\AttributeTok{type =} \StringTok{"multiplicative"}\NormalTok{)}
\FunctionTok{plot}\NormalTok{(Y.decomp, }\AttributeTok{lwd=}\DecValTok{2}\NormalTok{)}
\end{Highlighting}
\end{Shaded}

\includegraphics{Quantitative_Finance_Project_files/figure-latex/unnamed-chunk-12-1.pdf}

\emph{Summary Statistics of the Data}

\begin{Shaded}
\begin{Highlighting}[]
\FunctionTok{summary}\NormalTok{(Y)}
\end{Highlighting}
\end{Shaded}

\begin{verbatim}
##      Close       
##  Min.   : 37.06  
##  1st Qu.: 77.53  
##  Median :134.32  
##  Mean   :123.06  
##  3rd Qu.:160.55  
##  Max.   :197.57
\end{verbatim}

\begin{Shaded}
\begin{Highlighting}[]
\FunctionTok{describe}\NormalTok{(Y)}
\end{Highlighting}
\end{Shaded}

\begin{verbatim}
##    vars   n   mean    sd median trimmed   mad   min    max  range skew kurtosis
## X1    1 261 123.06 46.64 134.32   124.8 50.88 37.06 197.57 160.51 -0.4    -1.16
##      se
## X1 2.89
\end{verbatim}

\hypertarget{data-transformation-to-achieve-stationarity-1}{%
\subsubsection{Data Transformation To Achieve
Stationarity}\label{data-transformation-to-achieve-stationarity-1}}

Now, we will have to perform some data transformation to achieve
Stationarity.

\begin{Shaded}
\begin{Highlighting}[]
\CommentTok{\# Remove the trend from the data}
\CommentTok{\# First log Difference}
\NormalTok{d.Y }\OtherTok{=} \FunctionTok{diff}\NormalTok{(}\FunctionTok{log}\NormalTok{(Y)) }\CommentTok{\# to remove trend}
\FunctionTok{plot}\NormalTok{(d.Y, }\AttributeTok{main =} \StringTok{"Log of Apple weekly close prices time series"}\NormalTok{, }\AttributeTok{lwd=}\DecValTok{2}\NormalTok{)}
\end{Highlighting}
\end{Shaded}

\includegraphics{Quantitative_Finance_Project_files/figure-latex/unnamed-chunk-14-1.pdf}

\begin{Shaded}
\begin{Highlighting}[]
\NormalTok{d.logY}\FloatTok{.52} \OtherTok{=} \FunctionTok{diff}\NormalTok{(d.Y, }\AttributeTok{lag =} \DecValTok{52}\NormalTok{)  }\CommentTok{\# remove seasonality}
\FunctionTok{plot}\NormalTok{(d.logY}\FloatTok{.52}\NormalTok{, }\AttributeTok{main =}\StringTok{" Log of Apple weekly close price time series with lag = 52 "}\NormalTok{, }\AttributeTok{lwd=}\DecValTok{2}\NormalTok{, }\AttributeTok{col=}\StringTok{"blue"}\NormalTok{)}
\end{Highlighting}
\end{Shaded}

\includegraphics{Quantitative_Finance_Project_files/figure-latex/unnamed-chunk-14-2.pdf}

\begin{Shaded}
\begin{Highlighting}[]
\CommentTok{\# test for stationarity}
\FunctionTok{adf.test}\NormalTok{(d.logY}\FloatTok{.52}\NormalTok{)}
\end{Highlighting}
\end{Shaded}

\begin{verbatim}
## Warning in adf.test(d.logY.52): p-value smaller than printed p-value
\end{verbatim}

\begin{verbatim}
## 
##  Augmented Dickey-Fuller Test
## 
## data:  d.logY.52
## Dickey-Fuller = -5.5748, Lag order = 5, p-value = 0.01
## alternative hypothesis: stationary
\end{verbatim}

\hypertarget{comment-we-see-that-the-series-is-stationary-enough-to-do-any-kind-of-time-series-modelling.-1}{%
\subsubsection{Comment: We see that the series is stationary enough to
do any kind of time series
modelling.}\label{comment-we-see-that-the-series-is-stationary-enough-to-do-any-kind-of-time-series-modelling.-1}}

\hypertarget{identification-of-the-best-model-1}{%
\subsection{Identification of the best
model}\label{identification-of-the-best-model-1}}

\begin{Shaded}
\begin{Highlighting}[]
\CommentTok{\# Model Selection}
\FunctionTok{set.seed}\NormalTok{(}\DecValTok{123}\NormalTok{)}
\FunctionTok{auto.arima}\NormalTok{(d.logY}\FloatTok{.52}\NormalTok{, }\AttributeTok{stationary =} \ConstantTok{TRUE}\NormalTok{, }\AttributeTok{ic =} \FunctionTok{c}\NormalTok{(}\StringTok{"aic"}\NormalTok{), }\AttributeTok{trace =} \ConstantTok{TRUE}\NormalTok{)}
\end{Highlighting}
\end{Shaded}

\begin{verbatim}
## 
##  Fitting models using approximations to speed things up...
## 
##  ARIMA(2,0,2)(1,0,1)[52] with non-zero mean : Inf
##  ARIMA(0,0,0)            with non-zero mean : -580.0943
##  ARIMA(1,0,0)(1,0,0)[52] with non-zero mean : -674.1046
##  ARIMA(0,0,1)(0,0,1)[52] with non-zero mean : Inf
##  ARIMA(0,0,0)            with zero mean     : -581.9694
##  ARIMA(1,0,0)            with non-zero mean : -578.5567
##  ARIMA(1,0,0)(1,0,1)[52] with non-zero mean : -678.1467
##  ARIMA(1,0,0)(0,0,1)[52] with non-zero mean : Inf
##  ARIMA(0,0,0)(1,0,1)[52] with non-zero mean : -678.5859
##  ARIMA(0,0,0)(0,0,1)[52] with non-zero mean : Inf
##  ARIMA(0,0,0)(1,0,0)[52] with non-zero mean : -674.561
##  ARIMA(0,0,1)(1,0,1)[52] with non-zero mean : -678.4059
##  ARIMA(1,0,1)(1,0,1)[52] with non-zero mean : -676.8726
##  ARIMA(0,0,0)(1,0,1)[52] with zero mean     : -677.6064
## 
##  Now re-fitting the best model(s) without approximations...
## 
##  ARIMA(0,0,0)(1,0,1)[52] with non-zero mean : Inf
##  ARIMA(0,0,1)(1,0,1)[52] with non-zero mean : Inf
##  ARIMA(1,0,0)(1,0,1)[52] with non-zero mean : Inf
##  ARIMA(0,0,0)(1,0,1)[52] with zero mean     : Inf
##  ARIMA(1,0,1)(1,0,1)[52] with non-zero mean : Inf
##  ARIMA(0,0,0)(1,0,0)[52] with non-zero mean : -634.7971
## 
##  Best model: ARIMA(0,0,0)(1,0,0)[52] with non-zero mean
\end{verbatim}

\begin{verbatim}
## Series: d.logY.52 
## ARIMA(0,0,0)(1,0,0)[52] with non-zero mean 
## 
## Coefficients:
##          sar1     mean
##       -0.5661  -0.0030
## s.e.   0.0612   0.0024
## 
## sigma^2 = 0.002465:  log likelihood = 320.4
## AIC=-634.8   AICc=-634.68   BIC=-624.78
\end{verbatim}

\hypertarget{comment-auto-arima-indicates-us-that-the-best-arima-model-is-arima00010052}{%
\subsubsection{Comment: auto arima indicates us that the best arima
model is
ARIMA(0,0,0)(1,0,0){[}52{]}}\label{comment-auto-arima-indicates-us-that-the-best-arima-model-is-arima00010052}}

\hypertarget{weekly-model}{%
\subsection{Weekly Model}\label{weekly-model}}

\begin{Shaded}
\begin{Highlighting}[]
\DocumentationTok{\#\# Setting up the Model}
\NormalTok{Weekly.model }\OtherTok{\textless{}{-}} \FunctionTok{arima}\NormalTok{(d.logY}\FloatTok{.52}\NormalTok{, }\FunctionTok{c}\NormalTok{(}\DecValTok{0}\NormalTok{, }\DecValTok{0}\NormalTok{, }\DecValTok{0}\NormalTok{) , }\AttributeTok{seasonal =} \FunctionTok{list}\NormalTok{(}\AttributeTok{order =} \FunctionTok{c}\NormalTok{(}\DecValTok{1}\NormalTok{, }\DecValTok{0}\NormalTok{, }\DecValTok{0}\NormalTok{), }\AttributeTok{period =} \DecValTok{52}\NormalTok{))}
\end{Highlighting}
\end{Shaded}

\hypertarget{model-evaluation-1}{%
\subsection{Model evaluation}\label{model-evaluation-1}}

\begin{Shaded}
\begin{Highlighting}[]
\DocumentationTok{\#\# Diagnostic Checking}
\NormalTok{U}\FloatTok{.2} \OtherTok{\textless{}{-}}\NormalTok{ Weekly.model}\SpecialCharTok{$}\NormalTok{residuals}
\FunctionTok{plot}\NormalTok{(U}\FloatTok{.2}\NormalTok{)}
\end{Highlighting}
\end{Shaded}

\includegraphics{Quantitative_Finance_Project_files/figure-latex/unnamed-chunk-17-1.pdf}

\begin{Shaded}
\begin{Highlighting}[]
\CommentTok{\# Normality of the Residuals}
\FunctionTok{qqnorm}\NormalTok{(U}\FloatTok{.2}\NormalTok{); }\FunctionTok{qqline}\NormalTok{(U}\FloatTok{.2}\NormalTok{)}
\end{Highlighting}
\end{Shaded}

\includegraphics{Quantitative_Finance_Project_files/figure-latex/unnamed-chunk-17-2.pdf}

\begin{Shaded}
\begin{Highlighting}[]
\CommentTok{\# ACF of residuals}
\FunctionTok{acf}\NormalTok{(U}\FloatTok{.2}\NormalTok{)}
\end{Highlighting}
\end{Shaded}

\includegraphics{Quantitative_Finance_Project_files/figure-latex/unnamed-chunk-17-3.pdf}

\begin{Shaded}
\begin{Highlighting}[]
\FunctionTok{plot}\NormalTok{(}\FunctionTok{density}\NormalTok{(U}\FloatTok{.2}\NormalTok{))}
\end{Highlighting}
\end{Shaded}

\includegraphics{Quantitative_Finance_Project_files/figure-latex/unnamed-chunk-17-4.pdf}

\begin{Shaded}
\begin{Highlighting}[]
\CommentTok{\#Using tsdiag tools}
\FunctionTok{tsdiag}\NormalTok{(Weekly.model, }\AttributeTok{gof.lag =} \DecValTok{15}\NormalTok{)}
\end{Highlighting}
\end{Shaded}

\includegraphics{Quantitative_Finance_Project_files/figure-latex/unnamed-chunk-17-5.pdf}

\hypertarget{forecasting-1}{%
\subsection{Forecasting}\label{forecasting-1}}

Using our best model, we can forecast the observations for the next 2
years.

\begin{Shaded}
\begin{Highlighting}[]
\NormalTok{Weekly.pred }\OtherTok{\textless{}{-}} \FunctionTok{predict}\NormalTok{(Weekly.model, }\AttributeTok{n.ahead =} \DecValTok{2}\SpecialCharTok{*}\DecValTok{52}\NormalTok{)}
\FunctionTok{ts.plot}\NormalTok{(Y, }\FunctionTok{exp}\NormalTok{(Weekly.pred}\SpecialCharTok{$}\NormalTok{pred), }\AttributeTok{log =} \StringTok{"y"}\NormalTok{, }\AttributeTok{lty =}\FunctionTok{c}\NormalTok{(}\DecValTok{1}\NormalTok{,}\DecValTok{3}\NormalTok{), }\AttributeTok{lwd =} \DecValTok{1}\NormalTok{)}
\end{Highlighting}
\end{Shaded}

\includegraphics{Quantitative_Finance_Project_files/figure-latex/unnamed-chunk-18-1.pdf}

\newpage

\hypertarget{use-the-best-subsets-arima-approach-to-specify-a-model-for-the-monthly-close-data.}{%
\section{5. Use the best subsets ARIMA approach to specify a model for
the monthly close
data.}\label{use-the-best-subsets-arima-approach-to-specify-a-model-for-the-monthly-close-data.}}

\begin{Shaded}
\begin{Highlighting}[]
\CommentTok{\# Aggregate daily data to monthly}
\NormalTok{Monthly }\OtherTok{\textless{}{-}} \FunctionTok{to.monthly}\NormalTok{(Daily)}
\FunctionTok{colnames}\NormalTok{(Monthly) }\OtherTok{\textless{}{-}} \FunctionTok{c}\NormalTok{(}\StringTok{"Open"}\NormalTok{, }\StringTok{"High"}\NormalTok{, }\StringTok{"Low"}\NormalTok{, }\StringTok{"Close"}\NormalTok{, }\StringTok{"Volume"}\NormalTok{, }\StringTok{"Adjusted"}\NormalTok{)}

\CommentTok{\# Create monthly close prices time series}
\NormalTok{Z }\OtherTok{\textless{}{-}} \FunctionTok{ts}\NormalTok{(}\AttributeTok{data =}\NormalTok{ Monthly}\SpecialCharTok{$}\NormalTok{Close, }\AttributeTok{start =} \FunctionTok{c}\NormalTok{(}\DecValTok{2019}\NormalTok{, }\DecValTok{1}\NormalTok{, }\DecValTok{1}\NormalTok{), }\AttributeTok{end =}  \FunctionTok{c}\NormalTok{(}\DecValTok{2024}\NormalTok{, }\DecValTok{1}\NormalTok{, }\DecValTok{1}\NormalTok{), }\AttributeTok{frequency =} \DecValTok{12}\NormalTok{)}

\CommentTok{\# Create monthly close prices time series}
\FunctionTok{plot}\NormalTok{(Z, }\AttributeTok{main=}\StringTok{"Apple monthly close prices time series"}\NormalTok{, }\AttributeTok{lwd =} \DecValTok{2}\NormalTok{)}
\end{Highlighting}
\end{Shaded}

\includegraphics{Quantitative_Finance_Project_files/figure-latex/unnamed-chunk-19-1.pdf}

\hypertarget{decomposing-the-data-2}{%
\subsection{Decomposing the Data}\label{decomposing-the-data-2}}

Decomposing the data into its trend, seasonal, and random error
components will give some idea how these components relate to the
observed dataset.

\begin{Shaded}
\begin{Highlighting}[]
\CommentTok{\# Decomposition using multiplicative type }
\NormalTok{Z.decomp }\OtherTok{\textless{}{-}} \FunctionTok{decompose}\NormalTok{(Z, }\AttributeTok{type =} \StringTok{"multiplicative"}\NormalTok{)}

\FunctionTok{plot}\NormalTok{(Z.decomp, }\AttributeTok{lwd=}\DecValTok{2}\NormalTok{)}
\end{Highlighting}
\end{Shaded}

\includegraphics{Quantitative_Finance_Project_files/figure-latex/unnamed-chunk-20-1.pdf}

\emph{Summary Statistics of the Data}

\begin{Shaded}
\begin{Highlighting}[]
\FunctionTok{summary}\NormalTok{(Z)}
\end{Highlighting}
\end{Shaded}

\begin{verbatim}
##    Min. 1st Qu.  Median    Mean 3rd Qu.    Max. 
##   41.61   73.45  132.69  122.95  162.51  196.45
\end{verbatim}

\begin{Shaded}
\begin{Highlighting}[]
\FunctionTok{describe}\NormalTok{(Z)}
\end{Highlighting}
\end{Shaded}

\begin{verbatim}
##    vars  n   mean    sd median trimmed   mad   min    max  range  skew kurtosis
## X1    1 61 122.95 48.02 132.69  124.37 54.84 41.61 196.45 154.84 -0.35    -1.23
##      se
## X1 6.15
\end{verbatim}

\hypertarget{data-transformation-to-achieve-stationarity-2}{%
\subsubsection{Data Transformation To Achieve
Stationarity}\label{data-transformation-to-achieve-stationarity-2}}

Now, we will have to perform some data transformation to achieve
Stationarity.

\begin{Shaded}
\begin{Highlighting}[]
\CommentTok{\# Remove the trend from the data}
\CommentTok{\# First log Difference}
\NormalTok{d.Z }\OtherTok{=} \FunctionTok{diff}\NormalTok{(}\FunctionTok{log}\NormalTok{(Z)) }\CommentTok{\# to remove trend}

\FunctionTok{plot}\NormalTok{(d.Z, }\AttributeTok{main =} \StringTok{"Log of Apple monthly close prices time series"}\NormalTok{, }\AttributeTok{lwd=}\DecValTok{2}\NormalTok{)}
\end{Highlighting}
\end{Shaded}

\includegraphics{Quantitative_Finance_Project_files/figure-latex/unnamed-chunk-22-1.pdf}

\begin{Shaded}
\begin{Highlighting}[]
\NormalTok{d.logZ}\FloatTok{.6} \OtherTok{=} \FunctionTok{diff}\NormalTok{(d.Z, }\AttributeTok{lag =} \DecValTok{6}\NormalTok{)  }\CommentTok{\# remove seasonality}
\FunctionTok{plot}\NormalTok{(d.logZ}\FloatTok{.6}\NormalTok{, }\AttributeTok{main =}\StringTok{" Log of Apple monthly close price time series with lag = 6 "}\NormalTok{, }\AttributeTok{lwd=}\DecValTok{2}\NormalTok{, }\AttributeTok{col=}\StringTok{"blue"}\NormalTok{)}
\end{Highlighting}
\end{Shaded}

\includegraphics{Quantitative_Finance_Project_files/figure-latex/unnamed-chunk-22-2.pdf}

\begin{Shaded}
\begin{Highlighting}[]
\CommentTok{\# test for stationarity}
\FunctionTok{adf.test}\NormalTok{(d.logZ}\FloatTok{.6}\NormalTok{)}
\end{Highlighting}
\end{Shaded}

\begin{verbatim}
## 
##  Augmented Dickey-Fuller Test
## 
## data:  d.logZ.6
## Dickey-Fuller = -0.91431, Lag order = 3, p-value = 0.9434
## alternative hypothesis: stationary
\end{verbatim}

\hypertarget{comment-we-see-that-the-series-is-stationary-enough-to-do-any-kind-of-time-series-modelling.-2}{%
\subsubsection{Comment: We see that the series is stationary enough to
do any kind of time series
modelling.}\label{comment-we-see-that-the-series-is-stationary-enough-to-do-any-kind-of-time-series-modelling.-2}}

\hypertarget{identification-of-the-best-model-2}{%
\subsection{Identification of the best
model}\label{identification-of-the-best-model-2}}

\begin{Shaded}
\begin{Highlighting}[]
\CommentTok{\# Model Selection}
\FunctionTok{set.seed}\NormalTok{(}\DecValTok{123}\NormalTok{)}
\FunctionTok{auto.arima}\NormalTok{(d.logZ}\FloatTok{.6}\NormalTok{, }\AttributeTok{stationary =} \ConstantTok{TRUE}\NormalTok{, }\AttributeTok{ic =} \FunctionTok{c}\NormalTok{(}\StringTok{"aic"}\NormalTok{), }\AttributeTok{trace =} \ConstantTok{TRUE}\NormalTok{)}
\end{Highlighting}
\end{Shaded}

\begin{verbatim}
## 
##  ARIMA(2,0,2)(1,0,1)[12] with non-zero mean : Inf
##  ARIMA(0,0,0)            with non-zero mean : 4.418034
##  ARIMA(1,0,0)(1,0,0)[12] with non-zero mean : 7.33503
##  ARIMA(0,0,1)(0,0,1)[12] with non-zero mean : 6.529795
##  ARIMA(0,0,0)            with zero mean     : 3.424238
##  ARIMA(0,0,0)(1,0,0)[12] with non-zero mean : 5.730565
##  ARIMA(0,0,0)(0,0,1)[12] with non-zero mean : 5.782058
##  ARIMA(0,0,0)(1,0,1)[12] with non-zero mean : 7.573207
##  ARIMA(1,0,0)            with non-zero mean : 6.09048
##  ARIMA(0,0,1)            with non-zero mean : 5.347148
##  ARIMA(1,0,1)            with non-zero mean : 6.949911
## 
##  Best model: ARIMA(0,0,0)            with zero mean
\end{verbatim}

\begin{verbatim}
## Series: d.logZ.6 
## ARIMA(0,0,0) with zero mean 
## 
## sigma^2 = 0.06011:  log likelihood = -0.71
## AIC=3.42   AICc=3.5   BIC=5.41
\end{verbatim}

\hypertarget{comment-auto-arima-indicates-us-that-the-best-arima-model-is-arima000}{%
\subsubsection{Comment: auto arima indicates us that the best arima
model is
ARIMA(0,0,0)}\label{comment-auto-arima-indicates-us-that-the-best-arima-model-is-arima000}}

\hypertarget{weekly-model-1}{%
\subsection{Weekly Model}\label{weekly-model-1}}

\begin{Shaded}
\begin{Highlighting}[]
\DocumentationTok{\#\# Setting up the Model}
\NormalTok{Monthly.model }\OtherTok{\textless{}{-}} \FunctionTok{arima}\NormalTok{(d.logZ}\FloatTok{.6}\NormalTok{, }\FunctionTok{c}\NormalTok{(}\DecValTok{0}\NormalTok{, }\DecValTok{0}\NormalTok{, }\DecValTok{0}\NormalTok{) )}
\end{Highlighting}
\end{Shaded}

\hypertarget{model-evaluation-2}{%
\subsection{Model evaluation}\label{model-evaluation-2}}

\begin{Shaded}
\begin{Highlighting}[]
\DocumentationTok{\#\# Diagnostic Checking}
\NormalTok{U}\FloatTok{.3} \OtherTok{\textless{}{-}}\NormalTok{ Monthly.model}\SpecialCharTok{$}\NormalTok{residuals}
\FunctionTok{plot}\NormalTok{(U}\FloatTok{.3}\NormalTok{)}
\end{Highlighting}
\end{Shaded}

\includegraphics{Quantitative_Finance_Project_files/figure-latex/unnamed-chunk-25-1.pdf}

\begin{Shaded}
\begin{Highlighting}[]
\CommentTok{\# Normality of the Residuals}
\FunctionTok{qqnorm}\NormalTok{(U}\FloatTok{.3}\NormalTok{); }\FunctionTok{qqline}\NormalTok{(U}\FloatTok{.3}\NormalTok{)}
\end{Highlighting}
\end{Shaded}

\includegraphics{Quantitative_Finance_Project_files/figure-latex/unnamed-chunk-25-2.pdf}

\begin{Shaded}
\begin{Highlighting}[]
\CommentTok{\# ACF of residuals}
\FunctionTok{acf}\NormalTok{(U}\FloatTok{.3}\NormalTok{)}
\end{Highlighting}
\end{Shaded}

\includegraphics{Quantitative_Finance_Project_files/figure-latex/unnamed-chunk-25-3.pdf}

\begin{Shaded}
\begin{Highlighting}[]
\FunctionTok{plot}\NormalTok{(}\FunctionTok{density}\NormalTok{(U}\FloatTok{.3}\NormalTok{))}
\end{Highlighting}
\end{Shaded}

\includegraphics{Quantitative_Finance_Project_files/figure-latex/unnamed-chunk-25-4.pdf}

\begin{Shaded}
\begin{Highlighting}[]
\CommentTok{\#Using tsdiag tools}
\FunctionTok{tsdiag}\NormalTok{(Monthly.model, }\AttributeTok{gof.lag =} \DecValTok{15}\NormalTok{)}
\end{Highlighting}
\end{Shaded}

\includegraphics{Quantitative_Finance_Project_files/figure-latex/unnamed-chunk-25-5.pdf}
\#\# Forecasting Using our best model, we can forecast the observations
for the next 2 years.

\begin{Shaded}
\begin{Highlighting}[]
\NormalTok{Monthly.pred }\OtherTok{\textless{}{-}} \FunctionTok{predict}\NormalTok{(Monthly.model, }\AttributeTok{n.ahead =} \DecValTok{2}\SpecialCharTok{*}\DecValTok{12}\NormalTok{)}
\FunctionTok{ts.plot}\NormalTok{(Z, }\FunctionTok{exp}\NormalTok{(Monthly.pred}\SpecialCharTok{$}\NormalTok{pred), }\AttributeTok{log =} \StringTok{"y"}\NormalTok{, }\AttributeTok{lty =}\FunctionTok{c}\NormalTok{(}\DecValTok{1}\NormalTok{,}\DecValTok{3}\NormalTok{), }\AttributeTok{lwd =} \DecValTok{3}\NormalTok{)}
\end{Highlighting}
\end{Shaded}

\includegraphics{Quantitative_Finance_Project_files/figure-latex/unnamed-chunk-26-1.pdf}

\hypertarget{briefly-comment-on-the-differences-between-the-best-models-for-daily-weekly-and-monthly-data.}{%
\section{6. Briefly comment on the differences between the best models
for daily, weekly and monthly
data.}\label{briefly-comment-on-the-differences-between-the-best-models-for-daily-weekly-and-monthly-data.}}

\hypertarget{comparing-models}{%
\subsection{Comparing Models}\label{comparing-models}}

Looking at the AIC/BIC values, coefficients, and diagnostic statistics
of each of those three models, the Daily model in term of performance is
better than Weekly one which its turn is better than the Monthly one.

\end{document}
